\documentclass[11pt]{amsart}	%%%%%%%%
%%%%%%%%%%%%%%%%%%%%%%%%%%%%%%%%%%%%%%%%%%%%%%%%%%%%%%%%%%%%%%%%%%%%%%%%%%%%%%%%%%%%%%%%%%%%%%%%%%%%

%%%%%%%%%%%%%%%%%%%%%%%%%%%%%%%%%%%%%%%%%%%%%%%%%%%%%%%%%%%%%%%%%%%%%%%%%%%%%%%%%%%%%%%%%%%%%%%%%%%%
%%%%%%%%%%%%%%%%%%%%%%%%%%%%			Packages			%%%%%%%%%%%%%%%%%%%%%%%%%%%%%%%%%%%%%%%%
\usepackage[margin=2.9cm]{geometry}	%%%%														%%%%
\usepackage{amsthm}	%%%%%%%%%%%%%%%%%%%%														%%%%
\usepackage{amsmath}	%%%%%%%%%%%%%%%%														%%%%
\usepackage{amssymb}	%%%%%%%%%%%%%%%%														%%%%
\usepackage{enumerate}	%%%%%%%%%%%%%%%%														%%%%
\usepackage{comment}	%%%%%%%%%%%%%%%%														%%%%
\usepackage{multicol}	%%%%%%%%%%%%%%%%														%%%%
\usepackage{tabularx}	%%%%%%%%%%%%%%%%														%%%%
\usepackage{parskip}	%%%%%%%%%%%%%%%%														%%%%

%%%%%%%%%%%%%%%%%%%%%%%%%%%%%%%%%%%%%%%%%%%%%%%%%%%%%%%%%%%%%%%%%%%%%%%%%%%%%%%%%%%%%%%%%%%%%%%%%%%%
%%%%%%%%%%%%%%%%%%%%%%%%%%%%			Languages			%%%%%%%%%%%%%%%%%%%%%%%%%%%%%%%%%%%%%%%%
\usepackage[LGR,T1]{fontenc}	%%%%%%%%														%%%%
\usepackage[utf8]{inputenc}	%%%%%%%%%%%%														%%%%
\usepackage{lmodern}	%%%%%%%%%%%%%%%%														%%%%
\usepackage[greek,english]{babel}	%%%%														%%%%
\usepackage{alphabeta}	%%%%%%%%%%%%%%%%														%%%%


%%%%%%%%%%%%%%%%%%%%%%%%%%%%%%%%%%%%%%%%%%%%%%%%%%%%%%%%%%%%%%%%%%%%%%%%%%%%%%%%%%%%%%%%%%%%%%%%%%%%
%%%%%%%%%%%%%%%%%%%%%%%%%%%%			Title page			%%%%%%%%%%%%%%%%%%%%%%%%%%%%%%%%%%%%%%%%
\title{Teaching Statement}	%%%%%%%%%%%%														%%%%
\author{Vardakis Dimitris}	%%%%%%%%%%%%														%%%%
\address{Department of Mathematics, Michigan State University, East Lansing, Michigan 48824}	%%%%
\email{vardakis@msu.edu}	%%%%%%%%%%%%														%%%%
%\thanks{}	%%%%%%%%%%%%%%%%%%%%%%%%%%%%														%%%%

%%%%%%%%%%%%%%%%%%%%%%%%%%%%%%%%%%%%%%%%%%%%%%%%%%%%%%%%%%%%%%%%%%%%%%%%%%%%%%%%%%%%%%%%%%%%%%%%%%%%
%%%%%%%%%%%%%%%%%%%%%%%%%%%%			Setup				%%%%%%%%%%%%%%%%%%%%%%%%%%%%%%%%%%%%%%%%
%\setlength{\parindent}{0pt}

\begin{document}

\maketitle
%%%%%%%%%%%%%%%%%%%%%%%%%%%%%%%%%%%%%%%%%%%%%%%%%%%%%%%%%%%%%%%%%%%%%%%%%%%%%%%%%%%%%%%%%%%%%%%%%%%%

I have been teaching at Michigan State University (MSU) for the past 5 years at the position of a Graduate Teaching Assistant (GTA). Below, I explain my practices inside a classroom and present what experiences I have had so far related to teaching.

\section*{In-class Strategies}

Mathematics is a language, and as such what better way is there to teach than communication? I always try to lead my classes through discussion, asking questions, expecting answers, and as a general rule I make it clear that I am expecting responses from the students. Usually, I find it convenient to proceed into a proof (or into a new theorem or idea) step-by-step and ask the students what the next logical step and/or conclusion will be. Naturally, students might be hesitant to engage in the beginning, but as they see their peers participating, more and more people hop in the discussion. In order to expedite this, I also try to actively engage myself in discussions and interactions with the students that appear more shy or weak compared to the rest, and at the same time I pay attention not to have the same student(s) (typically the most advanced ones) be the only ones responding all the time.

Another strategy I follow is to have students do group work on the board. Of course, this is harder to do with bigger audiences, but I never cease to look for opportunities. This setup in a way forces the students to talk and communicate with one another without feeling the pressure of standing out since everyone is doing the exact same thing. In every class I've been able to implement this, the students' feedback was extremely positive with many of them saying they enjoyed the class and the group work and that it helped them understand the material better. Unfortunately, the bigger the class the harder it is to do have students on the board. Still I insist on having them work in groups even if it is from their desks.

A major factor when doing group work is the group's consistency. As a semester progresses ``teams'' tend to stabilise as students make friends with one another. However, I pay attention to assign the more advanced students with a different company every time and never together. This gives the opportunity to their peers to see things from an alternative point of view and additionally they are more eager to ask questions. Simultaneously, this allows the people who more advanced to exercise explaining notions and ideas to other people. In top of this, I have observed (as I like to oversee and interact with my groups) that groups with \emph{only one} advanced student tend to make more uniform progress towards their assignments compared to groups where all the people have a good understanding and feel more independent.

\section*{Experience in Teaching}

In my years as a GTA, I had the chance to teach several classes and lead the recitation of many more. I have worked on the courses of Differential Equations, Linear Algebra, and Transitions (a logic and number theory course). These are a mix of computational and proof based courses. I have worked with both the regular and the honours class of Differential Equations and Linear Algebra. I have also been assigned multiple times as the GTA for the graduate-level course of Complex Analysis.

The students in all my classes so far have been a mix of many different backgrounds. Most diverse being those of Differential Equations, I have worked with a crowd consisting of biologists, chemists, engineers, teachers, mathematicians, physicists, programmers, and some extra few \emph{at the same class}. As a result, I had to make accommodations to my explanations and examples based on this blend of backgrounds.

Needless to say, I have also taught several classes, and worked as a Lead Teaching Assistant (Lead TA) in an online setting during the COVID-19 pandemic's lockdowns.

\section*{Impact on MSU}

I have taken part in the MSU's mentoring programme, which is designed to help and guide new tutors through their first teaching experiences. The trainee enters a cycle of observation (of an experienced instructor), teaching (the class they observed), and getting feedback (from the instructor whose class they taught); this happens at least twice for each novice tutor. I have been the guiding instructor two times through this programme. Along the same lines, I have been assigned as a Lead TA (both in person and online) and I was supervising young tutors and other graduate students while assorting their tasks and responsibilities.

The policy of the Department of Mathematics does not allow GTA to decide on the curriculum of the courses. However, owning to my experience teaching Differential Equations, I contributed to reshaping this particular syllabus. In particular, pieces unrelated to the consequent material were removed, and certain chapter was shortened as it was covered extensively in another parallel course of the programme. With these changes the course focused more on actually solving differential equations and became more coherent.

\end{document}
